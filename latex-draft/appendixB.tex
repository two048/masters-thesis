\clearpage
\appendix
\phantomsection
\addcontentsline{toc}{chapter}{Appendix B}
\chapter*{Appendix B: Tables}

There are 51 entries for each of the sections mentioned below, which would take quite a number of pages to be printed, due to that reason we have tabulated the first 5 entries from each section, and have attached a link to the whole dataset, uploaded to the Google Drive (just like the notebooks) for future reference.

Link for the Dataset: \href{https://drive.google.com/drive/folders/1bzr7zFtRo2lOGgZHOYy6GC69jYzhsx8n?usp=drive_link}{https://drive.google.com/drive/folders/1bzr7zFtRo2lOG\\gZHOYy6GC69jYzhsx8n?usp=drive\_link}

\begin{figure}[h]
    \centering
    \includegraphics[width=0.2\linewidth]{images/data_sheets_qr.png}
\end{figure}

\singlespacing
\section*{Sampled Dataset}
\begin{longtable}{|c|p{6cm}|p{6cm}|}
\hline
\textbf{Section} & \textbf{English} & \textbf{Hindi} \\
\hline
7 & Sense of expression once explained.—Every expression which is explained in any part of this Code, is used in every part of this Code in conformity with the explanation. & {\hindifont एक बार स्पष्टीकॄत वाक्यांश का अभिप्राय - हर वाक्यांश, जिसका स्पष्टीकरण इस संहिता के किसी भाग में किया गया है, इस संहिता के हर भाग में उस स्पष्टीकरण के अनुरूप ही प्रयोग किया गया है।} \\
\hline
19 & “Judge”.—The word “Judge” denotes not only every person who is officially designated as a Judge, but also every person who is empowered by law to give, in any legal proceeding, civil or criminal, a definitive judgment, or a judgment which, if not appealed against, would be definitive, or a judgment which, if confirmed by some other authority, would be definitive, or who is one of a body or persons, which body of persons is empowered by law to give such a judgment. & {\hindifont न्यायाधीश - न्यायाधीश शब्द न केवल हर ऐसे व्यक्ति का द्योतक है, जो पद रूप से न्यायाधीश अभिहित हो, किन्तु उस हर व्यक्ति का भी द्योतक है, जो किसी क़ानूनी कार्यवाही में, चाहे वह सिविल हो या आपराधिक, अन्तिम निर्णय या ऐसा निर्णय, जो उसके विरुद्ध अपील न होने पर अन्तिम हो जाए या ऐसा निर्णय, जो किसी अन्य प्राधिकारी द्वारा पुष्ट किए जाने पर अन्तिम हो जाए, देने के लिए विधि द्वारा सशक्त किया गया हो, अथवा जो उस व्यक्ति निकाय में से एक हो, जो व्यक्ति निकाय ऐसा निर्णय देने के लिए विधि द्वारा सशक्त किया गया हो।} \\
\hline
25 & “Fraudulently”.—A person is said to do a thing fraudulently if he does that thing with intent to defraud but not otherwise. & {\hindifont कपटपूर्वक - कोई व्यक्ति किसी कार्य को कपट करने के आशय से करता है, उसे कपटपूर्वक कृत्य कहा जाता है, अन्यथा नहीं।} \\
\hline
37 & Co-operation by doing one of several acts constituting an offence.—When an offence is committed by means of several acts, whoever intentionally co-operates in the commission of that offence by doing any one of those acts, either singly or jointly with any other person, commits that offence. & {\hindifont कई कार्यों में से किसी एक कार्य को करके अपराध गठित करने में सहयोग करना। - जब कि कोई अपराध कई कार्यों द्वारा किया जाता है, तब जो भी कोई या तो अकेले या किसी अन्य व्यक्ति के साथ सम्मिलित होकर उन कार्यों में से कोई एक कार्य करके उस अपराध के किए जाने में साशय सहयोग करता है, तो वह उस अपराध को करता है।} \\
\hline
47 & “Animal”.—The word “animal” denotes any living creature, other than a human being. & {\hindifont जीवजन्तु - जीवजन्तु शब्द मानव से भिन्न किसी जीवधारी का द्योतक है ।} \\
\hline
\end{longtable}
\doublespacing

\section*{Translated Sample}

\singlespacing
\begin{tabular}{|c|p{6cm}|p{6cm}|}
\hline
\textbf{Section} & \textbf{Google Translate} & \textbf{Azure Translate} \\
\hline
7 & Meaning of a once clarified phrase - Every phrase which is explained in any part of this Code is used in every part of this Code in the same manner as that explanation. & Meaning of a once clarified phrase - Every phrase which is explained in any part of this Code is used in every part of this Code in the same manner as that explanation. \\
\hline
19 & Judge.—The word `Judge' denotes not only every person who is designated a judge by office, but also of every person  Who has been empowered by law to give a final judgment in any legal proceeding, whether civil or criminal, or a decision which becomes final when there is no appeal against it or a decision which becomes final when confirmed by any other authority, or who is one of the body of persons; A body of persons empowered by law to give such a decision. & The word `Judge' denotes not only every person who is designated a judge by office, but also every person who is empowered by law to give a final judgment in any legal proceeding, whether civil or criminal, or a decision which becomes final when there is no appeal against it or a decision which becomes final when confirmed by any other authority, or who is one of a body of persons empowered by law to give such a decision.  \\
\hline
25 & Fraudulently – A person does something with the intention of deceit, it is said to be an act of deception, not otherwise. & Fraudulently - When a person does something with the intent to deceive, it is said to be a fraudulent act and not otherwise. \\
\hline
37 & To cooperate in constituting an offense by doing any one of several tasks. - When an offense is committed by several acts, whoever intentionally cooperates in the commission of that offense by either alone or in association with any other person by doing any of those acts, commits that offense. & To cooperate in committing an offense by doing any one of several acts. - When an offense is committed by several acts, whoever intentionally cooperates in the commission of that offense by doing any one of those acts, either alone or in association with any other person, commits that offense. \\
\hline
47 & Fauna : The word animal denotes any living being other than human beings. & Animal - The word animal denotes any living creature other than a human being. \\
\hline
\end{tabular}

\section*{Metrics \& Scores}

\vspace{0.5cm}

\subsection*{Sentence Length \& BLEU Score}

\begin{center}
    \begin{table}[h]
    \setlength\extrarowheight{5pt}
    \centering
    \begin{tabular}{|c|c|c|c|c|c|}
        \hline
        \textbf{Section} & \textbf{reference\_len} & \textbf{google\_len} & \textbf{azure\_len} & \textbf{google\_bleu} & \textbf{azure\_bleu} \\
        \hline
        7 & 29 & 31 & 32 & 72.36 & 72.13 \\
        \hline
        19 & 83 & 55 & 86 & 52.68 & 78.06 \\
        \hline
        25 & 22 & 22 & 21 & 75.99 & 61.3 \\
        \hline
        37 & 49 & 47 & 49 & 84.35 & 87.98 \\
        \hline
        47 & 13 & 11 & 12 & 86.59 & 76.39 \\
        \hline
    \end{tabular}
\end{table}
\end{center}

\subsection*{ROUGE Score}

\begin{center}
    \begin{table}[h]
    \setlength\extrarowheight{5pt}
    \centering
    \begin{tabular}{|c|c|c|}
        \hline
        \textbf{Section} & \textbf{google\_rouge} & \textbf{azure\_rouge}\\
        \hline
        7 & [73.33, 51.72, 42.86, 73.33] & [75.41, 57.63, 56.14, 75.41] \\
        \hline
        19 & [73.91, 57.35, 47.76, 69.57] & [74.56, 47.9, 30.3, 69.82] \\
        \hline
        25 & [63.64, 28.57, 10.0, 45.45] & [46.51, 24.39, 10.26, 37.21] \\
        \hline
        37 & [83.33, 61.7, 52.17, 70.83] & [79.59, 60.42, 44.68, 63.27] \\
        \hline
        47 & [91.67, 81.82, 80.0, 91.67] & [80.0, 52.17, 38.1, 72.0] \\
        \hline
    \end{tabular}
\end{table}
\end{center}

\subsection*{Human Assigned Scores}

\begin{table}[h]
    \setlength\extrarowheight{5pt}
    \centering
    \begin{tabular}{|c|c|c|}
        \hline
        \textbf{Section} & \textbf{google\_hum\_score} & \textbf{azure\_hum\_score} \\
        \hline
        7 & 4 & 4 \\
        \hline
        19 & 3 & 3 \\
        \hline
        25 & 4 & 2 \\
        \hline
        37 & 4 & 3 \\
        \hline
        47 & 5 & 2 \\
        \hline
    \end{tabular}
\end{table}

\clearpage
\appendix